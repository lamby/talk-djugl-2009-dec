\documentclass{beamer}

% Minimalistic
\setbeamertemplate{navigation symbols}{}
\setbeamercolor{background canvas}{bg=black}
\setbeamercolor{normal text}{fg=white}
\setbeamercolor{structure}{fg=white}

% Nicer fonts
\usefonttheme{structurebold}
\setbeamerfont*{frametitle}{size=\normalsize,series=\bfseries}

\usepackage{listings} 

\newcommand{\LintExampleSized}[2]{
	\vskip 1em
	\lstinputlisting[frame=single,language=python,basicstyle=\ttfamily#1]{listings/lint-#2-input.txt}
	\vskip 0.5em
	\lstinputlisting[basicstyle=\tiny\ttfamily]{listings/lint-#2-output.txt}
}

\newcommand{\LintExample}[1]{
	\LintExampleSized{\footnotesize}{#1}
}

\title{django-lint}
\author{Chris Lamb}
\date{3rd December 2009}

\begin{document}

\frame{
	\titlepage
}

\frame {
	\tableofcontents
}

\setlength{\parskip}{1cm}

\section{Introduction to linting}
	\subsection{What is linting?}

	\frame {
		Linting is:

		\begin{itemize}
	 	\item For finding suspicious source code
		\item Examples of linters; {\tt pylint}, {\tt pychecker}, Coverity, most compilers, etc.
		\item {\it ``Hey, doesn't `./manage.py validate' do that?''}
		\end{itemize}
	}

	\frame {
		Linting is not:

		\begin{itemize}
		\item A panacea for code quality
		\item A replacement for unit tests
		\item An authority
		\item To be trusted
		\end{itemize}
	}

\section{django-lint}
	
	\frame {
		{\tt django-lint} is:

		\begin{itemize}
			\item A linter that finds suspicious code in Django projects
			\item Tries to promote the use of modular applications
			\item Current focus on suspicious data modelling
		\end{itemize}
	}

	\subsection{Examples}

	\frame {
		Example:

		\LintExample{01}

		{\it Hey, but sometimes X is a valid way to do that!}
	}

	\frame {
		\frametitle{Sub-optimal modelling}
		\LintExample{02}
	}

	\frame {
		\frametitle{Deprecated features}
		\LintExample{03}
	}

	\frame {
		\frametitle{Standard ordering of methods}
		\LintExample{04}
	}

	\frame {
		\frametitle{Default values}
		\LintExample{05}
	}

	\frame {
		\frametitle{Finding large models}
		\LintExample{06}
	}

	\frame {
		\frametitle{Problems in {\tt settings.py}}
		\LintExampleSized{\scriptsize}{07}
	}

	\frame {
		\vskip 0em
		{\it So, when should I be linting?}
	}

	\subsection{Implementation}
	\frame {
		\frametitle{Implementation of django-lint}
		\begin{itemize}
		\item Uses PyLint:
			\begin{itemize}
			\item Cool type and value inference 
			\item Hides (most of) the parsing AST nastiness
			\item Mechanisms for surpressing messagees
			\end{itemize}
		\item Just runs Django-specific tests -- append {\tt --pylint} for all
		\item Not a management command?
	 	\end{itemize}
	}

	\subsection{Future plans}
	\frame {
		\frametitle{Future plans}

		Project is never finished due to changes in:

		\begin{itemize}
		\item Django
		\item PyLint
		\item Fashion
		\end{itemize}
	}
	\frame {
		\frametitle{Future checks}

		More model checks:

		\begin{itemize}
		\item Default managers overriding getqueryset
		\item Other pet dislikes
		\end{itemize}
	}
	\frame {
		Views:

		\begin{itemize}
		\item Using convenience/portability wrappers ({\tt utils.hashcompat},
			{\tt utils.simplejson}, {\tt conf.settings} etc.)
		\item Using deprecated APIs (oldforms, etc.)
		\end{itemize}
	}
	\frame {
		Templates:

		\begin{itemize}
		\item Problematic - {\tt \{\% extends \%\}} and {\tt \{\% include \%\}} arguments are dynamic
		\item Eric Holscher's ``Gentlemans agreement on Django templates''
		\end{itemize}
	}
	\frame {
		Project layout:

		\begin{itemize}
		\item Forms not in {\tt myapp/forms.py}
		\item Admin classes not in {\tt myapp/admin.py}
		\end{itemize}
	}

\section{Conclusion}
	\subsection{Code}

	\frame {
		\frametitle{Getting the code}
		Code available under GPL-3+.

		{\tt \$ git clone git://git.chris-lamb.co.uk/django-lint }

		Tarballs at \url{http://chris-lamb.co.uk/projects/django-lint}

		In Debian as {\bf python-django-lint}.
	}

	\subsection{Questions}
	\frame {
		Questions?
	}

\end{document}

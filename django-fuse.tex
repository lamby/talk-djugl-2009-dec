\documentclass{beamer}

% Minimalistic
\setbeamertemplate{navigation symbols}{}
\setbeamercolor{background canvas}{bg=black}
\setbeamercolor{normal text}{fg=white}
\setbeamercolor{structure}{fg=white}

% Nicer title font
\setbeamerfont*{frametitle}{size=\normalsize,series=\bfseries}

\usepackage{graphicx}
\usepackage{verbatim} 

\usepackage{listings} 
\lstset{basicstyle=\ttfamily}

\title{django-fuse}
\author{Chris Lamb}
\date{3rd December 2009}

\begin{document}

\frame{\titlepage}

\section{FUSE and userspace filesystems}

	\subsection{What is FUSE?}

	\frame {
		\frametitle{"Filesystem in Userspace"}
		\begin{itemize}
			\item Implement filesystems in userspace programs instead of in the kernel
			\item No root privileges required
			\item Don't have to use C
			\item Implemented as kernel module and shared library
			\item MacFUSE?
		\end{itemize}
	}

	\frame {
		\frametitle{Examples of FUSE filesystems}
		\begin{itemize}
			\item FIXME
			\item FIXME
		\end{itemize}
	}

	\subsection{A typically \tt{python-fuse} app}
	\frame {
	 * Outline of the binding

		Request from OS	=>	python-fuse => def attr(path, arg1)
																				def stat(path, arg2)

		 - Maps a request for the operating system to function
		 - Request mapped to method corresponding to the data required
	}

	\frame {
		 Example
	}

\section{django-fuse}

	\frame {
		What it is

		 - improved API
		 - convenience
		 - ability to use
	}

	\subsection{Improved API}
	\frame {
	django-fuse is
	 * Abstraction over python-fuse
	 * Request path mapped to method, not the system call
		 - Using Django url resolution!
	 - Each part of your filesystem looks like a view
	 - Returns a 'Response' object depending on what you're modelling
			(eg. a file, directory, symlink, etc)
	}
	\subsection{Convenience}

	blah

	\subsection{Ability to use Django functionality}

	\frame {
		Same models, same.
	}

	\subsection{Examples}
	\frame {
		 * WrappedFiles
		 * Files with dynamic content
		 * Directory
		 * Database-backed files
		 * Dynamically named files
	}
	
	\subsection{Implementation}
	\frame {
	 * django-fuse implements each attr/stat/etc call behind the scenes
	 * When a request from the OS comes in:
			- Lookup view method using the path django.core.urlresolvers.resolve
			- Call the view and get the response back
			- Response objects actual implement the attr/stat details;
				 call which ever the OS is wanting this time
	}

\section{Conclusion}

	\subsection{Code}
	\frame {
		Code available under GPL-3+:

		\vskip 1em
		{\tt \$ git clone git://chris-lamb.co.uk/django-fuse }
		\vskip 1em

		Patches welcome.
	}

	\subsection{Questions}
	\frame {
			\vskip 0em
			Questions?
	}

\end{document}

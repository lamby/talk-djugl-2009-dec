\documentclass{beamer}

% Minimalistic
\setbeamertemplate{navigation symbols}{}
\setbeamercolor{background canvas}{bg=black}
\setbeamercolor{normal text}{fg=white}
\setbeamercolor{structure}{fg=white}

% Nicer fonts
\usefonttheme{structurebold}
\setbeamerfont*{frametitle}{size=\normalsize,series=\bfseries}

\usepackage{graphicx}
\usepackage{verbatim} 

\usepackage{listings} 
\lstset{basicstyle=\ttfamily\small,language=Python,frame=single,showspaces=false}

\title{django-fuse}
\author{Chris Lamb}
\date{3rd December 2009}

\begin{document}

\frame{
	\titlepage
}

\frame {
	\tableofcontents
}

\setlength{\parskip}{1cm}

\section{FUSE and userspace filesystems}

	\subsection{Filesytems}

	\frame {
		\frametitle{Filesystems}

		Filesystems "just" implement an API for manipulating files

		\pause

		\begin{itemize}
			\item Locally (ext2, ext3, NTFS) \pause
			\item Remotely (NFS, SMB) \pause
			\item Nowhere ({\tt /proc}) \pause
			\item Combination of the above
		\end{itemize}
	}

	\frame {
		\frametitle{Userspace filesystems}

		\pause

		$\rightarrow$ Filesystems not implemented in the kernel

		\pause

		\begin{itemize}
			\item Easier/faster to develop \pause
			\item No root privileges required \pause
			\item May not require C ({\tt python-fuse} etc.) \pause
			\item Some performance overhead
		\end{itemize}
	}

	\frame {
		\frametitle{FUSE}

		\pause

		Linux kernel module and shared library for writing {\bf F}ilesystems in {\bf USE}rspace.

		\pause

		\begin{itemize}
			\item SSHFS -- local access to remote files via SSH
			\item FlickrFS -- view/upload photos as regular files
			\item GmailFS -- uses your Gmail account as a storage medium
			\item WikipediaFS -- view and edit Wikipedia articles as regular files
			\item 10s more..
		\end{itemize}

		\pause

		{\it Why is this cool?}
	}

	\subsection{A typical {\tt python-fuse} app}
	\frame {
		\frametitle{Writing a filesystem with {\tt python-fuse}}

		\pause

		Well, "just" implement the API for manipulating files

		\pause

		Callback-based.

		Kernel $\leftrightarrow$ FUSE $\leftrightarrow$ Our filesystem
	}

	\frame {
		\frametitle{A typical {\tt python-fuse} app}

		\lstinputlisting{listings/python-fuse-1.txt}
	}

	\frame {
		\lstinputlisting{listings/python-fuse-2.txt}
	}

	\frame {
		\frametitle{}

		Request mapped to method corresponding to the {\bf system call}

		\pause

		\begin{itemize}
			\item Path management bit hairy \pause
			\item Methods really must return correct values \pause
			\item Even simple use-cases require a lot of code \pause
		\end{itemize}

		{\it It's like writing old-school web apps!}
	}

\section{django-fuse}

	\frame {
		\frametitle{How {\tt django-fuse} differs}

		\pause

		\begin{itemize}
			\item Handles paths better \pause
			\item Common error cases handled automatically \pause
			\item Convenience wrappers for common cases \pause
			\item MVC-like architecture for filesystems
		\end{itemize}
	}

	\frame {
		\frametitle{Path management in {\tt django-fuse}}

		\pause

		Still callback-based, but:

		\pause

		\begin{itemize}
			\item Kernel request is mapped to callback based on {\bf path}, not the system call
			\pause
			\item .. using Django's URL resolution!
			\pause
			\item Each part of filesystem looks like a Django view
		\end{itemize}
	}

	\subsection{Some examples}
	\frame {
		\frametitle{Example}

		{\tt settings.py}:

		\lstinputlisting{listings/django-fuse-1-1.txt}

		{\tt myfuseapp/urls.py}:

		\lstinputlisting{listings/django-fuse-1-2.txt}
	}

	\frame {
		\frametitle{Example (continued)}

		{\tt myfuseapp/views.py}:

		\lstinputlisting{listings/django-fuse-1-3.txt}

		\pause

		\lstinputlisting[language=sh]{listings/django-fuse-1-4.txt}
	}

	\frame {
		\frametitle{Example (continued)}

		Need to specify URL for the file we claim exists.

		{\tt myfuseapp/urls.py}:

		\lstinputlisting{listings/django-fuse-1-5.txt}
	}

	\frame {
		\frametitle{Example (continued)}

		{\tt myfuseapp/views.py}:

		\lstinputlisting{listings/django-fuse-1-6.txt}
	}

	\frame {
		\frametitle{Example (continued)}

		\lstinputlisting[language=sh]{listings/django-fuse-1-7.txt}
	}

	\frame {
		\frametitle{Wrapping an existing file}

		{\tt myfuseapp/urls.py}:

		\lstinputlisting{listings/django-fuse-2-1.txt}

		{\tt myfuseapp/views.py}:

		\lstinputlisting{listings/django-fuse-2-2.txt}
	}

	\frame {
		\frametitle{Wrapping an existing file (continued)}

		\lstinputlisting{listings/django-fuse-2-3.txt}
	}

	\frame {
		\frametitle{Dynamic files}

		{\tt myfuseapp/urls.py}:

		\lstinputlisting{listings/django-fuse-3-1.txt}

		{\tt myfuseapp/views.py}:

		\lstinputlisting{listings/django-fuse-3-2.txt}
	}

	\frame {
		\frametitle{Dynamic files (continued)}

		\lstinputlisting{listings/django-fuse-3-3.txt}
	}

	\frame {
		\frametitle{Other helpers}

		\pause

		\begin{itemize}
			\item Efficiently using {\tt QuerySet} objects in directory listings
			\item Symbolic links
		\end{itemize}
	}
	
	\subsection{Implementation}
	\frame {
		\frametitle{Implementation}

		\pause

		\begin{itemize}
		\item Implements every {\tt attr}/{\tt stat}/etc. call behind the scenes
		\pause
		\item When a request from the operating system arrives:
			\begin{itemize}
			\item Look up view method using {\tt django.core.urlresolvers.resolve}
			\pause	
			\item Call the view and get a response back
			\pause	
			\item The response objects implement the {\tt attr}/{\tt stat}/etc. details
			\pause	
			\item If we catch {\tt Http404}, return an appropriate error code for the request
			\end{itemize}
		\end{itemize}
	}

\section{Conclusion}

	\subsection{Other work}

	\frame {
		\frametitle{Other work}

		\pause

		\begin{itemize}
			\item MacFUSE
			\pause
			\item Plan9
		\end{itemize}
	}

	\subsection{Code}
	\frame {
		\frametitle{Getting the code}

		Code available under GPL-3+.

		{\tt \$ git clone git://chris-lamb.co.uk/django-fuse }

		Tarballs at \url{http://chris-lamb.co.uk/projects/django-fuse}

		Patches welcome.
	}

	\subsection{Questions}
	\frame {
		\vskip 0em
		Questions?
	}

\end{document}
